% This file is a valid PHP file and also a valid LaTeX file
% When processed with LaTeX, it will generate a blank template
% Loading with PHP will fill it with details

\documentclass{article}
% Required for proper escaping
\usepackage{textcomp} % Symbols

\usepackage[utf8]{inputenc}
\usepackage[T1]{fontenc} % Input format



\newcommand*{\plogo}{\fbox{$\mathcal{PL}$}} % Generic dummy publisher logo

\usepackage{xltxtra}
\XeTeXlinebreaklocale "th_TH"
\font\T="TH SarabunPSK:script=thai" at 14pt
\font\TB="TH SarabunPSK:script=thai" at 16pt
\font\E="Times New Roman:script=english" at 14pt

% Because Unicode etc. 
\usepackage{fontspec} % For loading fonts
\setmainfont{Liberation Serif} % Has a lot more symbols than Computer Modern
 
% Make placeholders visible
\newcommand{\placeholder}[1]{\textbf{$<$ #1 $>$}}

% Defaults for each variable
\newcommand{\collfullnum}{\placeholder{collfullnum}}
\newcommand{\counteachspec}{\placeholder{counteachspec}}
\newcommand{\countspecmonth}{\placeholder{countspecmonth}}
\newcommand{\monthDisplay}{\placeholder{monthDisplay}}
\newcommand{\pictureref}{\placeholder{pic_name}}





% LaTeX code for the invoice
\usepackage[landscape]{geometry}
 \geometry{
 a4paper,
 total={257mm,170mm},
 left=20mm,
 top=35mm,
 }
\usepackage{array}
\newcolumntype{L}[1]{>{\raggedright\let\newline\\\arraybackslash\hspace{0pt}}m{#1}}
\newcolumntype{C}[1]{>{\centering\let\newline\\\arraybackslash\hspace{0pt}}m{#1}}
\newcolumntype{R}[1]{>{\raggedleft\let\newline\\\arraybackslash\hspace{0pt}}m{#1}}
\usepackage{pdflscape}
\usepackage[svgnames,table]{xcolor}
\usepackage{ltablex}
\usepackage{graphicx}


\graphicspath{{D:/Apache24/htdocs/QedatamanagementRealestate/packet/LAYOUT-2/assets/duimage/}}
\definecolor{tableShade}{gray}{0.9}
\setlength{\parindent}{0pt}
\pagestyle{empty}


\title{ \textbf{\TB รายการพัสดุมีสถานะใช้งานได้ตามปกติ (ครุภัณฑ์)}}
\author{\begin{tabular}{m{1cm}m{2cm}m{3cm}m{3cm}m{2.5cm}m{2.5cm}m{2cm}m{5cm}}
 & & & & & & & \T ส่วนพิพิธภัณฑ์แมลง \\
 & & & & & & & \T สำนักวิจัยและอนุรักษ์ \\
\end{tabular}}
\date{}
\begin{document}
\maketitle


\begin{tabularx}{\linewidth}{|m{1cm}|m{4.0cm}|m{3.5cm}|m{3cm}|m{3.0cm}|m{2.5cm}|m{2cm}|m{2.0cm}|}
   
    \hline
	\centering\textbf{\T ลำดับ}  & \centering\textbf{\T รายการครุภัณฑ์}  &  \centering\textbf{\T หมายเลขครุภัณฑ์}  &  \centering\textbf{\T ผู้ใช้งาน} &  \centering\textbf{\T ผู้รับผิดชอบ} &  \centering\textbf{\T สถานที่} & \centering\textbf{\T รูป} & \textbf{\T หมายเหตุ} \\
	\endhead
    \hline

  
  %<?php																		/*
% */ foreach($data as $reportitem) {						/*   
 % */ ?>  /*
%<?php																		/*
	% */ 	echo "\n " . LatexTemplate::escape($reportitem['durable_seq']) . " & " .	/*	
	% */ 	" " ."\T ". LatexTemplate::escape($reportitem['durable_name_main']) . " & ".    /*	
	% */ 	" " ."\T ". LatexTemplate::escape($reportitem['durable_no_main']) . " & ".    /*
	% */ 	" " ."\T ". LatexTemplate::escape($reportitem['owner']) . " & ".    /*
	% */ 	" " ."\T ". LatexTemplate::escape($reportitem['response_name']) . " & ".    /*
    % */ 	" " ."\T ". LatexTemplate::escape($reportitem['place']) . " & ";	?>  
  %<?php																		/*	
% */  echo "\n" . "\\renewcommand{\\pictureref}{" . LatexTemplate::escape($reportitem['pic_name']) . "}\n"; ?>
\hspace{0.1\textwidth}
\begin{minipage}{0.9\textwidth}
\includegraphics[width=0.09\textwidth,height=0.09\textwidth]{\pictureref} 
\end{minipage}
\vspace{1mm}
&



\\
\hline
 %<?php	 /*
 % */ } ?>  /*
   

	\hline

	\end{tabularx}


	
\end{document}