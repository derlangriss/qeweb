%%%%
%
% ticket.sty example file for a pin for conferences
%
%%%%
% use the corresponding paper size for your ticket definition
\documentclass[letter,10pt]{letter}

% load ticket.sty with the appropriate ticket definition
\usepackage[qsbglabel,rowmode,crossmark]{ticket}%boxed, crossmark,circlemark,emptycrossmark,cutmark
\ticketNumbers{6}{26}
% load misc stuff
\usepackage{graphicx}
\usepackage{geometry}
 \geometry{
 letterpaper,
 total={170mm,257mm},
 left=20mm,
 top=20mm,
 }
%\usepackage{eso-pic}
\usepackage{calc}
%\usepackage{pst-barcode}
%\usepackage{auto-pst-pdf}
\usepackage{ragged2e}


\usepackage{textcomp} % Symbols
\usepackage[T1]{fontenc} % Input format

% Because Unicode etc.
\usepackage{fontspec} 
\setmainfont[Scale=1.25]{Arial} % Has a lot more symbols than Computer Modern
\linespread{1.3}\selectfont
\usepackage{qrcode}

%\makeatletter
%\ifcase \@ptsize \relax% 10pt
%  \newcommand{\miniscule}{\@setfontsize\miniscule{4}{5}}% \tiny: 5/6
%\or% 11pt
 %\newcommand{\miniscule}{\@setfontsize\miniscule{5}{6}}% \tiny: 6/7
%\or% 12pt
% \newcommand{\miniscule}{\@setfontsize\miniscule{5}{6}}% \tiny: 6/7
%\fi
%\makeatother

\renewcommand{\ticketdefault}{%
}

% now what do you like to put in your ticket
\newcommand{\qsbglabelitem}[1]{\ticket{%
    \put(0.6,4.0){#1}
}}

% Make placeholders visible
\newcommand{\placeholder}[1]{\textbf{$<$ #1 $>$}}
% Defaults for each variable
\newcommand{\specimenfullnumber}{\placeholder{specimenfullnumber}}
\newcommand{\tordername}{\placeholder{tordername}}
\newcommand{\familyname}{\placeholder{familyname}}
\newcommand{\genusname}{\placeholder{genusname}}
\newcommand{\speciesname}{\placeholder{speciesname}}

% LaTeX code for the invoice

\begin{document}
\RaggedRight
%\miniscule
\bfseries
\fontsize{3}{3}\selectfont
%\sffamily
%<?php																		/*
	% */ foreach($data as $labelitem) {						/*
      
	
	
        
      % */  echo "\n" . "\\renewcommand{\\specimenfullnumber}{" . LatexTemplate::escape($labelitem['specimenfullnumber']) . "}\n".    /*
      % */       "\n" . "\\renewcommand{\\tordername}{" . LatexTemplate::escape($labelitem['tordername']) . "}\n".    /*
      % */       "\n" . "\\renewcommand{\\familyname}{" . LatexTemplate::escape($labelitem['familyname']) . "}\n".    /*
      % */       "\n" . "\\renewcommand{\\genusname}{" . LatexTemplate::escape($labelitem['genusname']) . "}\n".    /*     
      % */       "\n" . "\\renewcommand{\\speciesname}{" . LatexTemplate::escape($labelitem['speciesname']) . "}\n"; ?>
      
      


\qsbglabelitem{ 
\parbox[r][8mm][c]{8mm}{
 \vspace{-3mm}
 \hspace{0mm}
\qrcode[level=M,height=0.7cm]{\specimenfullnumber}




 }
 \parbox[l][10mm][t]{24mm}{\specimenfullnumber{}\\ORDER: \tordername{}\\FAMILY: \familyname{}\\GENUS: \genusname{}\\SPECIES: \speciesname{}}

 }
 
            %<?php	 /*
 
 % */ } ?>  
 


\end{document}
