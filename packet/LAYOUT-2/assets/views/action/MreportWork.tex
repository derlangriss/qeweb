\documentclass{article}
\usepackage[usenames,dvipsnames]{xcolor}
\usepackage{tcolorbox}
\usepackage{tabularx}
\usepackage{array}
\usepackage{colortbl}

% Make placeholders visible
\newcommand{\placeholder}[1]{\textbf{$<$ #1 $>$}}

% Defaults for each variable
\newcommand{\collfullnum}{\placeholder{collfullnum}}
\newcommand{\counteachspec}{\placeholder{sompong}}
\newcommand{\countspecmonth}{\placeholder{countspecmonth}}
\newcommand{\monthDisplay}{\placeholder{monthDisplay}}
\newcommand{\yearDisplay}{\placeholder{yearDisplay}}




\tcbuselibrary{skins}

\newcolumntype{Y}{>{\raggedleft\arraybackslash}X}

\tcbset{tab1/.style={fonttitle=\bfseries\large,fontupper=\normalsize\sffamily,
colback=yellow!10!white,colframe=red!75!black,colbacktitle=Salmon!40!white,
coltitle=black,center title,freelance,frame code={
\foreach \n in {north east,north west,south east,south west}
{\path [fill=red!75!black] (interior.\n) circle (3mm); };},}}

\tcbset{tab2/.style={enhanced,fonttitle=\bfseries,fontupper=\normalsize\sffamily,
colback=yellow!10!white,colframe=red!50!black,colbacktitle=Salmon!40!white,
coltitle=black,center title}}




% LaTeX code for produce QSBGinsects label
\begin{document}




\begin{tcolorbox}[tab2,tabularx={X||Y},title=Report Of the Month,boxrule=0.5pt]
Collection Number & Number of Specimens          \\\hline\hline



%<?php                                    /*
% */ foreach($data as $reportitem) {           /*
% */  echo "\n" . "\\renewcommand{\\collfullnum}{" . LatexTemplate::escape($reportitem['collfullnum']) . "}\n".    /*
% */       "\n" . "\\renewcommand{\\countspecmonth}{" . LatexTemplate::escape($reportitem['countspecmonth']) . "}\n"; ?>
\collfullnum{}   & \counteachspec{}  \\

%<?php /* % */ } ?>  












\end{tcolorbox}



\end{document}