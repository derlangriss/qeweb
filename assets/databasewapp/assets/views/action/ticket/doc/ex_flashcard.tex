%%%
%
% ticket.sty example file for flashcards for word learning
%
%%%%
% use the corresponding paper size for your ticket definition
\documentclass[a4paper,10pt]{letter}

\usepackage[flashCard,boxed]{ticket}
\usepackage{graphicx}  % load misc stuff
\usepackage{syntonly}  %to test without output
\usepackage{fancybox}

% make your default ticket. \ticketdefault is somewhat like a background
\renewcommand{\ticketdefault}{%no background
}

% now what do you like to put in your ticket
\newcommand{\card}[2]{\ticket{%
    \put(32.5,15){\makebox[0mm]{\centering{\huge{\textbf{#1}}}}}
 }}

\newcommand{\entryroot}[2]{}

\newcommand{\raiz}[3]{\ticket{%
    \put(2,31){{$\sqrt{\enspace}$}}
    \put(32.5,15){\makebox[0mm]{{\huge{\textit{{ #1 }}}}}}
}}

%puts a bullet on the word.
\newcommand{\blt}{$\bullet$\enspace}

%command for the synonyms
\newcommand{\syn}[1]{
  \begin{center}
     \fbox{\parbox[t]{50mm}{\centering{#1}}}
  \end{center}
}

%command for the related words.
\newcommand{\relwords}[1]{$\mathbf{\sim}$ #1}

%command for the examples
\newcommand{\example}[1]{\\[1.5mm]\textit{#1}}




%% you can generate this part from a database!
\begin{document}
\raiz{a}{without}{
  \entryroot{amoral}{neither moral nor inmoral}
  \entryroot{anonymous}{of unknown authorship or origin}
  \entryroot{atrophy}{the wasting away of body tissue}
}
\raiz{ab/abs}{off, away from, apart, down}{
  \entryroot{abduct}{to take by force}
  \entryroot{abhor}{to hate detest}
  \entryroot{abdicate}{renbounce of relinquish a throne}
  \entryroot{abstinence}{forbearance from any indulgence of appetite}
}
\card{abate}{to lessen to subside}
\card{abdication}{giving up control authority}
\card{aberration}{straying away from what is normal}
\card{abet}{help, aid\\\blt Act as a complice\\\blt encourage smb
  (in doing wrong)} 
\card{abeyance}{Expectancy\\\blt suspended action}
\card{abhor}{to hate to detest}
\card{abide}{Stay, dwell\\\blt be faithful to; endure\\\blt continue, bear}
\card{abjure}{promise or swear to give up\\\blt reject, abandon}
\card{abrogate}{repeal or annul by authority}
\card{abscond}{to go away suddenly (to avoid arrest)\\\blt depart
  secretly \example{The patron ABSCONDED from the restaurant without
    paying the bill}}
\card{abstruse}{difficult to comprehend obscure}
\card{abut}{Touch, be in contact with\\\blt touch along a border}
\card{abysmal}{bottomless extreme}
\raiz{ac/arc}{sharp, bitter}{
  \entryroot{acid}{something that is sharp, sour, or ill natured}
  \entryroot{acute}{sharp at the end}
  \entryroot{acerbic}{sour or astringent in taste, harsh in temper}
  \entryroot{exacerbate}{to increase in bitterness or violenceM
    aggravate}
  \entryroot{acrid}{sharp or biting to the taste or smell}
  \entryroot{acrimonious}{caustic, stinging, or bitter}
}
\card{acclaimed}{welcomed with shouts and approval}
\card{accolade}{praise, approval\\\blt Rite to mark the recognition of
a merit\\\blt confer knighthood}
\card{accretion}{Growth (esp organic).  The growing of separate things
  into one\\\blt Concretion} 
\raiz{act/ag}{to do; to drive; to force}{
  \entryroot{agile}{quick and well coordinated}
  \entryroot{agitate}{}
  \entryroot{litigate}{to make the subject of a lawsuit}
  \entryroot{prodigal}{wastfully or recklessly extravagant}
  \entryroot{pedagogue}{a teacher}
  \entryroot{synagoge}{a gathering or congregation of Jews}
}
\raiz{ad/al}{to, toward, near}{
  \entryroot{adapt, adjacent, addict, admire,\\ address, adhere}{}
  \entryroot{adjoin}{to be close or in contact with}
  \entryroot{advocate}{to plead in favour of}
}
\card{adamant}{kind of stone\\\blt inflexible, obdurate, unyielding}
\card{admonitory}{containing warning}
\card{advocate}{speech in favour of}
\card{adorn}{add beauty decorate}
\card{adulteration}{making unpure poorer in quality}
\card{affable}{polite and friendly}
\card{affinity}{close connection relationship}
\card{aggravate}{make worse irritate}
\card{agile}{active quick-moving}
\raiz{al/ali/alter}{other, another}{
  \entryroot{alternative}{}
  \entryroot{alias}{}
  \entryroot{alibi}{the defense by an accused person that he was
    verificably elsewhere at the time of the crime}
  \entryroot{alien}{�ne born in another country; a foreigner}
  \entryroot{alter ego}{the second self, a sustitute or deputy}
  \entryroot{altruist}{concerned with the welfare of others}
}
\card{alacrity}{celerity\\\blt eager and cheerful readiness}
\card{allegiance}{Loyalty to one's king\\\blt Relation of feudal
  vassal to his superior  \syn{Fidelity, loyalty, support}}  
\card{alleviate}{make (pain) easier to bear}
\card{allay}{assuage}
\card{alloy}{Mixing of metals\\\blt To debase by mixing with something inferior}
\card{aloof}{reserved indifferent detached}
\raiz{am}{love}{
  \entryroot{amateur}{}
  \entryroot{amatory, enamored, enamorata}{}
  \entryroot{amenity}{agreeable ways or manners}
  \entryroot{amity}{friendship, peaceful harmony}
  \entryroot{amiable}{having or showing aggreable personal qualities}
  \entryroot{amicable}{characterized by exhibiting good will}
}
\card{amalgamate}{mix combine unite societies}
\raiz{amb}{to go, walk}{
  \entryroot{ambient}{moving freely, circulating}
  \entryroot{ambitious, preamble, ambulance}{}
  \entryroot{ambassador}{an authorized messenger or representative}
  \entryroot{ambulatory}{of, pertaining to, or capable of walking}
  \entryroot{ambush}{the act of lying concealed so as to attack by
    surprise}
}
\raiz{amb/amph}{both, more than one, around}{
  \entryroot{ambiguous}{open to various interpretations}
  \entryroot{amppibian}{}
  \entryroot{ambidextrous}{}
}
\card{ambidextrous}{able to use the left hand or the right equally
 well}
\card{ambiguous}{doubtful, uncertain}
\card{ambivalent}{having both of two contrary meanings}
%
%%% Local Variables: 
%%% mode: plain-tex
%%% TeX-master: t
%%% End:
% change the commands for the backside
\renewcommand{\card}[2]{\ticket{%
    \put(3,30){\parbox{58mm}{\large{\textbf{ #1:}}}}%
    \put(4,23){\parbox[t]{58mm} {\small {\blt #2}}}%
 }}

\renewcommand{\entryroot}[2]{\small{\textbf{#1 }}\textit{\scriptsize{{#2}}}$\|$}

\renewcommand{\raiz}[3]{\ticket{%
    \put(3,30){\parbox{58mm}{\textbf{#1:} #2}}%
    \put(3,13){\parbox{58mm}{#3}}%
 }}

\backside%
\oddsidemargin=8pt
\raiz{a}{without}{
  \entryroot{amoral}{neither moral nor inmoral}
  \entryroot{anonymous}{of unknown authorship or origin}
  \entryroot{atrophy}{the wasting away of body tissue}
}
\raiz{ab/abs}{off, away from, apart, down}{
  \entryroot{abduct}{to take by force}
  \entryroot{abhor}{to hate detest}
  \entryroot{abdicate}{renbounce of relinquish a throne}
  \entryroot{abstinence}{forbearance from any indulgence of appetite}
}
\card{abate}{to lessen to subside}
\card{abdication}{giving up control authority}
\card{aberration}{straying away from what is normal}
\card{abet}{help, aid\\\blt Act as a complice\\\blt encourage smb
  (in doing wrong)} 
\card{abeyance}{Expectancy\\\blt suspended action}
\card{abhor}{to hate to detest}
\card{abide}{Stay, dwell\\\blt be faithful to; endure\\\blt continue, bear}
\card{abjure}{promise or swear to give up\\\blt reject, abandon}
\card{abrogate}{repeal or annul by authority}
\card{abscond}{to go away suddenly (to avoid arrest)\\\blt depart
  secretly \example{The patron ABSCONDED from the restaurant without
    paying the bill}}
\card{abstruse}{difficult to comprehend obscure}
\card{abut}{Touch, be in contact with\\\blt touch along a border}
\card{abysmal}{bottomless extreme}
\raiz{ac/arc}{sharp, bitter}{
  \entryroot{acid}{something that is sharp, sour, or ill natured}
  \entryroot{acute}{sharp at the end}
  \entryroot{acerbic}{sour or astringent in taste, harsh in temper}
  \entryroot{exacerbate}{to increase in bitterness or violenceM
    aggravate}
  \entryroot{acrid}{sharp or biting to the taste or smell}
  \entryroot{acrimonious}{caustic, stinging, or bitter}
}
\card{acclaimed}{welcomed with shouts and approval}
\card{accolade}{praise, approval\\\blt Rite to mark the recognition of
a merit\\\blt confer knighthood}
\card{accretion}{Growth (esp organic).  The growing of separate things
  into one\\\blt Concretion} 
\raiz{act/ag}{to do; to drive; to force}{
  \entryroot{agile}{quick and well coordinated}
  \entryroot{agitate}{}
  \entryroot{litigate}{to make the subject of a lawsuit}
  \entryroot{prodigal}{wastfully or recklessly extravagant}
  \entryroot{pedagogue}{a teacher}
  \entryroot{synagoge}{a gathering or congregation of Jews}
}
\raiz{ad/al}{to, toward, near}{
  \entryroot{adapt, adjacent, addict, admire,\\ address, adhere}{}
  \entryroot{adjoin}{to be close or in contact with}
  \entryroot{advocate}{to plead in favour of}
}
\card{adamant}{kind of stone\\\blt inflexible, obdurate, unyielding}
\card{admonitory}{containing warning}
\card{advocate}{speech in favour of}
\card{adorn}{add beauty decorate}
\card{adulteration}{making unpure poorer in quality}
\card{affable}{polite and friendly}
\card{affinity}{close connection relationship}
\card{aggravate}{make worse irritate}
\card{agile}{active quick-moving}
\raiz{al/ali/alter}{other, another}{
  \entryroot{alternative}{}
  \entryroot{alias}{}
  \entryroot{alibi}{the defense by an accused person that he was
    verificably elsewhere at the time of the crime}
  \entryroot{alien}{�ne born in another country; a foreigner}
  \entryroot{alter ego}{the second self, a sustitute or deputy}
  \entryroot{altruist}{concerned with the welfare of others}
}
\card{alacrity}{celerity\\\blt eager and cheerful readiness}
\card{allegiance}{Loyalty to one's king\\\blt Relation of feudal
  vassal to his superior  \syn{Fidelity, loyalty, support}}  
\card{alleviate}{make (pain) easier to bear}
\card{allay}{assuage}
\card{alloy}{Mixing of metals\\\blt To debase by mixing with something inferior}
\card{aloof}{reserved indifferent detached}
\raiz{am}{love}{
  \entryroot{amateur}{}
  \entryroot{amatory, enamored, enamorata}{}
  \entryroot{amenity}{agreeable ways or manners}
  \entryroot{amity}{friendship, peaceful harmony}
  \entryroot{amiable}{having or showing aggreable personal qualities}
  \entryroot{amicable}{characterized by exhibiting good will}
}
\card{amalgamate}{mix combine unite societies}
\raiz{amb}{to go, walk}{
  \entryroot{ambient}{moving freely, circulating}
  \entryroot{ambitious, preamble, ambulance}{}
  \entryroot{ambassador}{an authorized messenger or representative}
  \entryroot{ambulatory}{of, pertaining to, or capable of walking}
  \entryroot{ambush}{the act of lying concealed so as to attack by
    surprise}
}
\raiz{amb/amph}{both, more than one, around}{
  \entryroot{ambiguous}{open to various interpretations}
  \entryroot{amppibian}{}
  \entryroot{ambidextrous}{}
}
\card{ambidextrous}{able to use the left hand or the right equally
 well}
\card{ambiguous}{doubtful, uncertain}
\card{ambivalent}{having both of two contrary meanings}
%
%%% Local Variables: 
%%% mode: plain-tex
%%% TeX-master: t
%%% End:
\end{document}








