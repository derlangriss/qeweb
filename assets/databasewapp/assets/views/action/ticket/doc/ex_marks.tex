%%%%
%
% ticket.sty example file for empty cross-marks
%
%%%%
% use the corresponding paper size for your ticket definition
\documentclass[a4paper,10pt]{letter}

% load ticket.sty with the appropriate ticket definition
\usepackage[freepin3,emptycrossmark]{ticket}

% load misc stuff
\usepackage{graphicx}
\usepackage{color}
%\usepackage{layout}

% make your default ticket. \ticketdefault is somewhat like a background
% here it is empty, because we use different tickets for the front an
% the back of our card
\renewcommand{\ticketdefault}{%
}

% now what do you like to put in your ticket
\newcommand{\acard}[1]{\ticket{%
    \put( 7, 20){\large Hello!}%
    \put( 7, 10){Number:\bfseries #1}%
}}

% a counter, which makes life easier...
\newcounter{numcards}

\begin{document}
%\layout\newpage
\sffamily

% cards with an empty crossmark
\whiledo{\thenumcards<25}{\stepcounter{numcards}\acard{\thenumcards}}%

% cards with cutmarks
\setcounter{numcards}{0}\ticketreset\clearpage%
\makeatletter\@emptycrossmarkfalse\@cutmarktrue\makeatother
\whiledo{\thenumcards<25}{\stepcounter{numcards}\acard{\thenumcards}}%

% cards boxed
\setcounter{numcards}{0}\ticketreset\clearpage%
\makeatletter\@cutmarkfalse\@boxedtrue\makeatother
\whiledo{\thenumcards<25}{\stepcounter{numcards}\acard{\thenumcards}}%

% cards with crossmarks
\setcounter{numcards}{0}\ticketreset\clearpage%
\makeatletter\@boxedfalse\@crossmarktrue\makeatother
\whiledo{\thenumcards<25}{\stepcounter{numcards}\acard{\thenumcards}}%

% cards with circlemarks
\setcounter{numcards}{0}\ticketreset\clearpage%
\makeatletter\@crossmarkfalse\@circlemarktrue\makeatother
\whiledo{\thenumcards<25}{\stepcounter{numcards}\acard{\thenumcards}}%


\end{document}


