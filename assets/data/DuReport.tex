% This file is a valid PHP file and also a valid LaTeX file
% When processed with LaTeX, it will generate a blank template
% Loading with PHP will fill it with details

\documentclass{article} 
% Required for proper escaping
\usepackage{textcomp} % Symbols

\usepackage[utf8]{inputenc}
\usepackage[T1]{fontenc} % Input format



\newcommand*{\plogo}{\fbox{$\mathcal{PL}$}} % Generic dummy publisher logo

\usepackage{xltxtra}
\usepackage{polyglossia}
\XeTeXlinebreaklocale "th"
\XeTeXlinebreakskip = 0pt plus 1pt


\fontspec[
ItalicFont={TH SarabunPSK Italic},
BoldFont={TH SarabunPSK Bold},
BoldItalicFont={TH SarabunPSK Bold Italic},
]{TH SarabunPSK}
\fontsize{18pt}{24pt}\selectfont
% Because Unicode etc. 


 
% Make placeholders visible
\newcommand{\placeholder}[1]{\textbf{$<$ #1 $>$}}

% Defaults for each variable
\newcommand{\collfullnum}{\placeholder{collfullnum}}
\newcommand{\counteachspec}{\placeholder{counteachspec}}
\newcommand{\countspecmonth}{\placeholder{countspecmonth}}
\newcommand{\monthDisplay}{\placeholder{monthDisplay}}
\newcommand{\pictureref}{\placeholder{pic_name}}





% LaTeX code for the invoice.
\usepackage[usegeometry]{typearea}% load before geometry
\usepackage[landscape]{geometry}
 \geometry{
 a4paper,
 total={257mm,170mm},
 left=20mm,
 top=35mm,
 }
\usepackage{array}
\newcolumntype{L}[1]{>{\raggedright\let\newline\\\arraybackslash\hspace{0pt}}m{#1}}
\newcolumntype{C}[1]{>{\centering\let\newline\\\arraybackslash\hspace{0pt}}m{#1}}
\newcolumntype{R}[1]{>{\raggedleft\let\newline\\\arraybackslash\hspace{0pt}}m{#1}}
\usepackage{pdflscape}
\usepackage[svgnames,table]{xcolor}
\usepackage{ltablex}
\usepackage{graphicx}
\usepackage{setspace}


\graphicspath{{D:/Apache24/htdocs/QedatamanagementRealestate/packet/LAYOUT-2/assets/duimage/}}
\definecolor{tableShade}{gray}{0.9}
\setlength{\parindent}{0pt}
\usepackage{fancyhdr}
\pagestyle{fancy}
\fancyhf{} % clear all header and footer fields 
\renewcommand{\headrulewidth}{0pt}

\fancyhead[C]{\begin{tabular}[t]{@{}p{10cm}@{}}
  \hfill \textbf{รายการพัสดุมีสถานะใช้งานได้ตามปกติ (ครุภัณฑ์)} \hfill \strut \\
  \\
\end{tabular}}
\fancyhead[R]{\begin{tabular}[t]{@{}>{\raggedleft}p{5cm}@{}}
   \\
   \hfill ส่วนพิพิธภัณฑ์แมลง \hfill \\
   \hfill สำนักวิจัยและอนุรักษ์ \hfill \\
\end{tabular}}
\fancyfoot[RO,LE]{\begin{tabular}[t]{@{}p{5cm}@{}}
 \\
   \hfill ลงชื่อ...............................................  \hfill \strut \\
   \\
   \hfill (นางส่งศรี โกวิทเทวาวงศ์ )  \hfill \strut \\
   \hfill ผู้รับผิดชอบ  \hfill \strut \\
\end{tabular}}

\setdefaultlanguage{thai}
\newfontfamily{\thaifont}[Script=Thai,
  SizeFeatures={Size=14},
  BoldFeatures={SizeFeatures={Size=14}}, 
  UprightFeatures={SizeFeatures={Size=14}},
  ItalicFeatures={SizeFeatures={Size=14}},
  BoldFeatures={SizeFeatures={Size=14}},
  BoldItalicFeatures={SizeFeatures={Size=14}}]{TH SarabunPSK}
\newcommand{\boldfont}[1]{{%
  \fontsize{10pt}{12pt}\normalfont #1%
}}
\setlength{\headheight}{35pt}
\begin{document}
\newgeometry{hmargin=2.0cm,bottom=45mm,height=145mm,includehead}
\fancyheadoffset{0pt}% recalculate headwidth for fancyhdr

\begin{tabularx}{\linewidth}{|m{1cm}|m{4.5cm}|m{3.5cm}|m{3cm}|m{3.0cm}|m{2.5cm}|@{}m{1.82cm}@{}|m{2.5cm}|}
   
    \hline
  \centering\textbf{ลำดับ}  & \centering\textbf{รายการครุภัณฑ์}  &  \centering\textbf{หมายเลขครุภัณฑ์}  &  \centering\textbf{ผู้ใช้งาน} &  \centering\textbf{ผู้รับผิดชอบ} &  \centering\textbf{สถานที่} & \centering\textbf{รูป} & \textbf{หมายเหตุ} \\
  \endhead
    \hline

  
  %<?php                                    /*
% */ foreach($data as $reportitem) {            /*   
 % */ ?>  /*
%<?php                                    /*
  % */  echo "\n " . LatexTemplate::escape($reportitem['durable_seq']) . " & " .  /*  
  % */  " " . LatexTemplate::escape($reportitem['durable_name_main']) . " & ".    /*  
  % */  " " . LatexTemplate::escape($reportitem['durable_no_main']) . " & ".    /*
  % */  " " . LatexTemplate::escape($reportitem['owner']) . " & ".    /*
  % */  " " . LatexTemplate::escape($reportitem['response_name']) . " & ".    /*
    % */  " " . LatexTemplate::escape($reportitem['place']) . " & ";  ?>  
    \vspace{-3.2mm}
  %<?php                                    /*  
% */  echo "\n" . "\\renewcommand{\\pictureref}{" . LatexTemplate::escape($reportitem['pic_name']) . "}\n"; ?>
\begin{minipage}{0.15\textwidth}
\includegraphics[width=0.5\textwidth,height=0.5\textwidth]{\pictureref} 
\end{minipage}
\vspace{-1.5mm}
\hspace*{-25cm}%
&



\\
\hline
 %<?php  /*
 % */ } ?>  /*
   

  \hline

  \end{tabularx}


  
\end{document}